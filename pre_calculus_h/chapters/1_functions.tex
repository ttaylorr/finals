\chapter{Functions and Graphs}

\section{Slope}
Slope defines the average rate of change on a function in a given interval,
\(\Delta x\) (this expression is defined as $x_2-x_1$).

Consider a function $f(x)$.  $f$ takes in x-coordinates, and spits out
corresponding y-coordinates, creating a set of ordered pairs that produces a
line, curve, or otherwise.

The formal definition of slope is given as follows:
\begin{equation}
m=\frac{f(x_2)-f(x_1)}{x_2-x_1}
\end{equation}

\subsection{Point-slope form}
Given the coordinate pair of some point in the x-y plane, and the slope of any
arbitrary line $m$, we can develop an equation for the line that has the slope
$m$ and goes through the coordinate $(x, y)$.

The general equation of that line is given as follows:
\begin{equation}
y-y_1=m(x-x_1)
\end{equation}

\subsection{Sketching lines}
The general equation of a linear line is as follows:
\begin{equation}
y=mx+b
\end{equation}

Therefore, the y-intercept is given when $x=0$.  To solve for the y-intercept of
the line given above, we evaluate it, plugging $0$ in for $x$, producing a
coordinate pair of $(0, f(0))$.

To sketch a line, preform the following steps:
\begin{enumerate}
\item{Find the coordinate of the y-intercept of the graph, according to the rule
given above}
\item{Sketch that location into the coordinate plane}
\item{Construct the two adjacent points on the plane according to the slope $m$
and the definition of slope}
\item{Connect the dots}
\end{enumerate}

\subsection{Properties of Slope}
Two lines are parallel iff the following condition is met: $m_1=m_2$.

\section{Functions}
A function $f$ maps a set of x-coordinates ($A$) to a set of y-coordinates
($B$).  Each element in $A$ is mapped to \textit{exactly one} point from the set
$B$.

\begin{itemize}
\item{Each element of $A$ must be matched to an element of $B$}
\item{Some elements of $B$ may not be matched with any elements of $A$}
\item{Two or more elements of $A$ may be matched with the same element $B$}
\item{An element of $A$ can not be matched with more than one distinct element
of $B$}
\end{itemize}

The difference quotient is defined as follows, and becomes important later:
\begin{equation}
\frac{f(x+h)-f(x)}{h}, h \ne 0
\end{equation}

Some formal definitions follow:
\begin{description}
\item[Function]{a relationship between two variables such that to each value of
the independent variable there is exactly one value of the dependent variable}
\item[Function notation]{$y=f(x)$}
\item[Domain]{the set of values $x$ for which $f$ can produce real values}
\item[Range]{the set of values that $f$ can produce for all given $x$ in the
domain}
\item[Implied domain]{the set of real numbers that satisfy the expression for
which $f$ is defined upon}
\end{description}

\section{Graphs of functions}
A function can be defined on any given interval as one of three things:
increasing, decreasing, or constant. The classifications of each follow:

For any interval $x_1$, $x_2$...
\begin{itemize}
\item The function is increasing for all $x$ such that $f(x_1)<f(x_2)$
\item The function is decreasing for all $x$ such that $f(x_1)>f(x_2)$
\item The function is constant for all $x$ such that $f(x_1)=f(x_2)$
\end{itemize}

A function has a relative minimum when there exists an interval $(x_1,x_2)$ such
that there is an $a$ that $x_1<x<x_2$ and $f(a) \leq f(x)$.

Conversly, a function has a relative maximum when there exists an interval
$(x_1,x_2)$ such that there is an $a$ that $x_1<x<x_2$ and $f(a) \geq f(x)$.

\subsection{Even and Odd functions}
\begin{itemize}
\item A function is even such that for all $x$ in the domain of $f$,
$f(-x)=f(x)$
\item A function is odd such that for all $x$ in the domain of $f$,
$f(-x)=-f(x)$
\end{itemize}

\section{Translations}
There are several types of translations that can be applied to parent functions
and what will become their "children".  For a parent function $f$, you can
produce a child function $g$ such that $g$ is defined in terms of $f(x)$.  The
simple translations follow:

\begin{enumerate}
\item Vertical shift $c$ units upwards: $g(x)=f(x)+c$
\item Vertical shift $c$ units downwards: $g(x)=f(x)-c$
\item Horizontal shift $c$ units to the left: $g(x)=f(x+c)$
\item Horizontal shift $c$ units to the right: $g(x)=f(x-c)$
\end{enumerate}

To reflect a function across an axis, take the following advice:

\begin{enumerate}
\item A reflection in the $x$-axis for a function $f$: $g(x)=-f(x)$
\item A reflection in the $y$-axis for a function $f$: $g(x)=f(-x)$
\end{enumerate}

To stretch or shirnk a function $g(x)$ in terms of $f(x)$, simply multiply $f$
by a scalar $c$.

\section{Combinations of Functions}
There are several ways that you can combine two functions $f$ and $g$ (for
example) to produce a new, composite function of the two:

\begin{enumerate}
\item Sum: $(f+g)(x)=f(x)+g(x)$
\item Difference: $(f-g)(x)=f(x)-g(x)$
\item Product: $(fg)(x)=f(x)*g(x)$
\item Quotient: $\frac{f}{g}(x)=\frac{f(x)}{g(x)}$
\end{enumerate}

\subsection{Composite Functions}
A composite function of $f$ and $g$ is defined as follows: $f(g(x))$.
Therefore, the range of $g$ must be limited to the domain of $f$ so that no $x$
values can be passed to the composite function that are invalid.

\section{Inverse Functions}
Two functions $f$, and $g$ are considered to be inverses of each other, iff both
of the following two conditions are met:

\begin{enumerate}
\item $f(g(x))=x$ for all $x$ in the domain of $g(x)$.
\item $g(f(x))=x$ for all $x$ in the domain of $f(x)$.
\end{enumerate}

To find an inverse function algebraically, preform the following steps:

\begin{enumerate}
\item Use the Horizontal Line Test to determine if $f$ has an inverse ($f^-1$)
  \begin{itemize}
  \item (If it doesn't, limit the domain of $f$ so that it does.)
  \end{itemize}
\item{Plug in $f^-1(x)$ into the original function and isolate}
\end{enumerate}
