\chapter{Analytical Trigonometry}
\section{Fundamental Identities}
\subsection{Reciprocal Identities}
\begin{itemize}
  \item{$sin(\theta)=\frac{1}{csc(\theta)}$}
  \item{$cos(\theta)=\frac{1}{sec(\theta)}$}
  \item{$tan(\theta)=\frac{1}{cot(\theta)}$}
  \item{$csc(\theta)=\frac{1}{sin(\theta)}$}
  \item{$sec(\theta)=\frac{1}{cos(\theta)}$}
  \item{$cot(\theta)=\frac{1}{tan(\theta)}$}
\end{itemize}

\subsection{Quotient Identities}
\begin{itemize}
  \item{$tan(\theta)=\frac{sin(\theta)}{cos(\theta)}$}
  \item{$cot(\theta)=\frac{cos(\theta)}{sin(\theta)}$}
\end{itemize}

\subsection{Pythagorean Identities}
\begin{itemize}
  \item{$sin^2(u)+cos^(u)=1$}
  \item{$1+tan^2(u)=sec^2(u)$}
  \item{$1+cot^2(u)=csc^2(u)$}
\end{itemize}

\subsection{Cofunction Identities}
\begin{itemize}
  \item{$sin(\frac{\pi}{2}-u)=cos(u)$}
  \item{$cos(\frac{\pi}{2}-u)=sin(u)$}
  \item{$tan(\frac{\pi}{2}-u)=cot(u)$}
  \item{$cot(\frac{\pi}{2}-u)=tan(u)$}
  \item{$sec(\frac{\pi}{2}-u)=csc(u)$}
  \item{$csc(\frac{\pi}{2}-u)=sec(u)$}
\end{itemize}

\section{Sum and Difference Formulae}
\begin{itemize}
  \item{$\sin(u \pm v)=\sin(u)\cos(v) \pm \cos(u)\sin(v)$}
  \item{$\cos(u \pm v)=\cos(u)\cos(v) \mp \sin(u)\sin(v)$}
  \item{$\tan(u \pm v)=\frac{\tan(u) \pm \tan(v)}{1 \mp \tan(u)\tan(v)}$}
\end{itemize}

\section{Double Angle Formulae}
\begin{itemize}
  \item{$\sin2u=2\sin u\cos u$}
  \item{$\cos2u=$ ...}
    \begin{enumerate}
      \item{$\cos^2u-\sin^2u$}
      \item{$2\cos^2u-1$}
      \item{$1-2\sin^2u$}
    \end{enumerate}
  \item{$\tan2u=\frac{2\tan u}{1-\tan^2y}$}
\end{itemize}

\section{Power Reducing Formulae}
\begin{itemize}
  \item{$\sin^2u=\frac{1-\cos 2u}{2}$}
  \item{$\cos^2u=\frac{1+\cos 2u}{2}$}
  \item{$\tan^2u=\frac{1-\cos 2u}{1+\cos 2u}$}
\end{itemize}

\section{Half Angle Formulae}
\begin{itemize}
  \item{$\sin(\frac{u}{2})=\pm\sqrt{\frac{1-\cos u}{2}}$}
  \item{$\cos(\frac{u}{2})=\pm\sqrt{\frac{1+\cos u}{2}}$}
  \item{$\tan(\frac{u}{2})=\frac{1-\cos u}{\sin u}=\frac{\sin u}{1+\cos u}$}
\end{itemize}

\section{Notes on these Formulae}
To use these formulas, it is important to know them well.  Look for patterns and
be able to expand them into simpler ones.  Once you have broken them down far
enough, it will be easy to solve for them, since you can evaluate expressions
inside of the functions easily.

\section{Law of Sines and Cosines}
Consider any triangle with sides $a$, $b$, and $c$, and angles opposite to their
respective sides, $A$, $B$, and $C$.  Given this triangle, we can make two
generalizations about it, both of which are supersets of the Pythagorean
Formula:

\subsection{The Law of Sines}
\begin{equation}
  \frac{\sin A}{a}=\frac{\sin B}{b}=\frac{\sin C}{c}
\end{equation}

\subsection{The Law of Cosines}
\begin{itemize}
  \item{$c^2=a^2+b^2-2ab\cos C$}
  \item{$a^2=b^2+c^2-2bc\cos A$}
  \item{$b^2=a^2+c^2-2ac\cos B$}
\end{itemize}
