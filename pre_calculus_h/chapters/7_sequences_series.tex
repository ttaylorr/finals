\chapter{Sequences and Series}
\section{Sequences and Series}
An infinite sequence is the sequence whose domain lines is defined on all
$x\in(0,\infty)$.  It is defined as follows:
\begin{equation}
a_1,a_2,a_3,...,a_n,...
\end{equation}

If the sequence were to only contain the first $n$ terms, it would be a
\textit{finite sequence}.

\section{Summation Notation}
To define the series that we expressed above, we can use a notation known as
"summation notation", which involves the use of the Greek letter
\textit{sigma}.  Let's define the sequence that we did earlier, this time using
summation notation:

\begin{equation}
\sum_{i=1}^na_i=a_1+a_2+a_3+a_4+...+a_n+...
\end{equation}

\subsection{Properties of Sums}
\begin{equation}
\sum_{i=1}^nc=cn
\end{equation}

\begin{equation}
\sum_{i=1}^nca_i=c\sum_{i=1}^{n}a_i
\end{equation}

\begin{equation}
\sum_{i=1}^{n}(a_i \pm b_i)=\sum_{i=1}^{n}a_i\pm\sum_{i=1}^{n}b_i
\end{equation}

\subsection{A Distinction}
There is an important distinction to be made: The first $n$ terms of any
sequence is referred to as the finite series, or a partial sum of teh entier
sequence, and is denoted by the following expression:
\begin{equation}
\sum_{i=1}^{n}a_i=a_1+a_2+a_3+a_4+...+a_n
\end{equation}

However, the infinite sries is all of the terms in an infinite sequence, and is
expressed by the following:
\begin{equation}
\sum_{i=1}^{\infty}a_i=a_1+a_2+a_3+a_4+...+a_n+...
\end{equation}

Consider the following two examples.  The first is a partial sum, and the second
is the sum of a series:
\begin{align*}
\sum_{i=1}^{3}\frac{3}{10^i}=\frac{3}{10^1}+\frac{3}{10^2}+\frac{3}{10^3}\\
=\frac{3}{10}+\frac{3}{100}+\frac{3}{1000}\\
=0.3+0.03+0.003\\
=0.333
\end{align*}

\begin{align*}
\sum_{i=1}^{\infty}\frac{3}{10^i}=\frac{3}{10^1}+\frac{3}{10^2}+\frac{3}{10^3}+...
\\
=\frac{3}{10}+\frac{3}{10}+\frac{3}{100}+...\\
=0.3+0.03+0.003+...\\
=0.333...
=\frac{1}{3}.
\end{align*}

\section{Arithmetic Sequences}
An arithmetic sequence is one where the differences between consecutive terms
are the same.  The sequence takes the following progression:
\begin{equation}
  a_1,a_1+d,a_1+2d,a_1+3d...
\end{equation}

\subsection{$n$-th term}
Therefore, the formula for the $n$th term is:
\begin{equation}
  a_1+(n-1)d
\end{equation}

\subsection{$n$-th partial sum}
To calculuate the $n$-th partial sum of the series, use the following formula:
\begin{equation}
  S_n=\frac{n}{2}(a_1+a_n)
\end{equation}

\section{Geometric Sequences}
A series is geometric when each consecutive is related by a ratio to the
previous term.  Consider the following example:
\begin{equation}
  a_1,a_1r,a_1r^2,...,a_1r^{n-1}
\end{equation}

\subsection{$n$-th term}
Therefore, the $n$-th term is defined as follows:
\begin{equation}
  a_n=a_1r^{n-1}
\end{equation}

\subsection{Summations of Finite Series}
The sum of a finite geometric series is given as follows:
\begin{equation}
  S_n=\sum_{i=1}^{n}a_1r^{i-1}=a_1\Big(\frac{1-r^n}{1-r}\Big)
\end{equation}

\subsection{Summations of Infinite Series}
The sum of an infinite geometric series is given as follows:
\begin{equation}
  S=\sum_{i=1}^{\infty}a_1r^i=\frac{a_1}{1-r}
\end{equation}
