\chapter{Trigonometric Functions}
\section{Radians}
A radian is a measure of the central angle $\theta$ such that the arc $s$
intercepted by the angle is equal to the length of the radius, $r$.

One radian is the measure of a central angle $\theta$ that intercepts an arc $s$
equal to the length of the radis $r$ of the circle: ($\theta$ is measured in
radians)
\begin{equation}
  \theta=\frac{s}{r}
\end{equation}

There are a few key terms that are used when describing angles inside of a
circle:
\begin{description}
  \item[Initial side]{The starting position of the ray}
  \item[Terminal side]{The other side of the ray}
  \item[Vertex]{The intersection of the initial and terminal side}
\end{description}

\subsection{Converting beteen radians and degrees}
\begin{enumerate}
  \item{To convert degrees to radians, simply multiply degrees by
    $\frac{\pi\text{rad}}{180}$}
  \item{To convert radians to degrees, multiply radians by
    $\frac{180}{\pi\text{rad}}$}
\end{enumerate}

\subsection{Linear and Angular Speed}
Arc legth is defined for a length $s$ as the quantity of the length of the
radius, $r$ and the central angle that $s$ is intercepted by, $\theta$:
\begin{equation}
  s=r\theta
\end{equation}

Therefore, linear speed is defined as the quotient of the arc-length and the
time that has passed:
\begin{equation}
  \text{linear speed}=\frac{\text{arc length}}{\text{time}}=\frac{s}{t}
\end{equation}

And as follows, angular speed is the same formula, except we are looking at
radians per second, as in the following equation:
\begin{equation}
  \text{angular speed}=\frac{\text{central angle}}{\text{time}}=\frac{\theta}{t}
\end{equation}

\section{Unit Circle Trigonometry}
Consider a unit circle (a circle of radius $1$ centered around $(0,0)$) defined
by the following equation: $x^2+y^2=1$.

For any point \textit{on} the circle, the coordinate pair that will intersect
the edge of the circle is defined as $P(\theta)=(cos(\theta),sin(\theta))$.

\begin{itemize}
  \item{$sin(t)=y$}
  \item{$cos(t)=x$}
  \item{$tan(t)=\frac{y}{x}$}
  \item{$csc(t)=\frac{1}{y}$}
  \item{$sec(t)=\frac{1}{x}$}
  \item{$cot(t)=\frac{x}{y}$}
\end{itemize}

\subsection{Even and Odd Trigonometric Functions}
\begin{itemize}
  \item{Even:}
    \begin{itemize}
      \item{$cos(-t)=cos(t)$}
      \item{$sec(-t)=sec(t)$}
    \end{itemize}
  \item{Odd:}
    \begin{itemize}
      \item{$sin(-t)=-sin(t)$}
      \item{$csc(-t)=-csc(t)$}
      \item{$tan(-t)=-tan(t)$}
      \item{$cot(-t)=-cot(t)$}
    \end{itemize}
\end{itemize}

\section{Right Triangle Trigonmetry}
Consider a right triangle.  A geometric figure with three sides, two of which
meet at a $90\deg$ angle.  The other side is reffered to as the hypoteneuse, and
the length of that side is equal to $\sqrt{a^2+b^2}$.

There are six trigonometric functions that we will study that relate to right
triangles.  They are defined as follows:
\begin{itemize}
  \item{$sin(\theta)=\frac{\text{opp}}{\text{hyp}}$}
  \item{$cos(\theta)=\frac{\text{adj}}{\text{hyp}}$}
  \item{$tan(\theta)=\frac{\text{opp}}{\text{adj}}$}
  \item{$csc(\theta)=\frac{1}{sin(\theta)}$}
  \item{$sec(\theta)=\frac{1}{cos(\theta)}$}
  \item{$cot(\theta)=\frac{1}{tan(\theta)}$}
\end{itemize}

\subsection{Trigonometric Identities}

\subsubsection{Reciprocal Identities}
This set of identities defines the relationships between pairs of trigonometric
functions as inverses of each other.

\begin{itemize}
  \item{$sin(\theta)=\frac{1}{csc(\theta)}$}
  \item{$cos(\theta)=\frac{1}{sec(\theta)}$}
  \item{$tan(\theta)=\frac{1}{cot(\theta)}$}
  \item{$csc(\theta)=\frac{1}{sin(\theta)}$}
  \item{$sec(\theta)=\frac{1}{cos(\theta)}$}
  \item{$cot(\theta)=\frac{1}{tan(\theta)}$}
\end{itemize}

\subsubsection{Quotient Identites}
This set of identnties defines two trigonometric functions as quotients of other
pairs of trigonometric functions.

\begin{itemize}
  \item{$tan(\theta)=\frac{sin(\theta)}{cos(\theta)}$}
  \item{$cot(\theta)=\frac{cos(\theta)}{sin(\theta)}$}
\end{itemize}

\section{Graphs of Sine and Cosine Functions}
The sine function is defined for all $x\in(-\infty,\infty)$, and has a range of
$[-1,1]$.  It has a period of $2\pi$ and has $x$-intercepts at $(n\pi, 0)$ and
$y$-intercepts at the origin.  It is an odd-function with original symmetry.

On the other hand, the cosine is also defined for all $x\in(-\infty,\infty)$,
and has a range of $[-1,1]$.  It has a period of $2\pi$ and has $x$-intercepts
at $(\frac{\pi}{2}+n\pi,0)$.  It has a $y$-intercept at $(1,0)$ and is an even
function with $y$-axis symmetry.

\subsection{Amplitude}
The amplitude of a function is defined as one half of the distance between its
relative minimum and maximum points.  For any trigonmetric function of the form
$y=a\text{sin}(x)$ or $y=a\text{cos}(x)$, the amplitude is $|a|$.

\subsection{Period}
The period of a function is the time it takes the function to repeat itself, for
any point.  The period itself is not a function, it is a constant value for all
points \textit{across} a function.  For any function that is defined by the
pattern: $y=a\text{sin}(x)$ or $y=a\text{cos}(x)$, the period is defined by the
following equation:
\begin{equation}
  \text{Period}=\frac{2\pi}{b}
\end{equation}

\section{Graphs of other Trigonometric Functions}

\subsection{The Tangent and its Inverse}
The graph of the tangent function is defined for all real numbers $x$, such that
$x\neq\frac{\pi}{2}+n\pi$.  Its range is defined for all
$x\in(-\infty,\infty)$ and has $x$-intercepts at $(n\pi,0)$, and a
$y$-intercept at the origin.  It has vertical asymptotes at
$x=\frac{\pi}{2}+n\pi$ and is an odd function with origianl symmetry.

The function $f(x)=cot(x)$ is exactly the $tan(x)$ function, but shifted
$\frac{\pi}{2}$ units to the right.  That is to say that
$f(x)=cot(x)=tan(x)+\frac{\pi}{2}$.

\subsection{The Cosecant and Secant Functions}
These functions are very easy to reason about.  Take their inverse functions,
and draw plot them on a graph.  Define vertical asymptotes when the origianl
function is $f(x)=0$, and draw points at the relative minimum and maximum across
the original function.  Stretch those points to the asymptotes through
continuous curves, and you have the function that you are trying to sketch.

The same rules apply when applying transformations to these functions.

\section{Inverse Trignonometric Functions}
The inverse sine function, or $f(x)=arcsin(x)$ is defined over the interval
$[-\frac{\pi}{2},\frac{\pi}{2}]$.  Simply take the sine function at those two
endpoints and plot the inverse function.  Apply transformations as normal.

\subsection{Characteristics of Inverse Trigonometric Functions}
\begin{description}
  \item[$f(x)=arcsin(x)$]{Defined over all $x\in[-1,1]$; $-\frac{\pi}{2} \leq
    f(x) \leq \frac{pi}{2}$}
  \item[$f(x)=arccos(x)$]{Defined over all $x\in[-1,1]$; $0 \leq f(x) \leq \pi$}
  \item[$f(x)=arctan(x)$]{Defined over all $x\in(-\infty,\infty)$;
    $-\frac{\pi}{2} \leq f(x) \leq \frac{\pi}{2}$}
\end{description}
