\chapter{Introductory Calculus}
\section{The Newton Quotient}
Consider any arbitrary curve.  At any point on that curve, there is a line which
is tangent to that curve.  How might we go about finding the slope of such a
line.  Well, if you consider two points, one some units to the left of the
point we are targeting, and the other some arbitrary amount of units to the
right, between those two points is a secant line.  If you were to take the limit
of the distance between the point we are targeting and the ones that we have
placed to the left and right of it, we would intuitively approach the slope of
the line tangent to the curve at that point.

This process is summarized in the Newton quotient, which is drawn out below:
\begin{equation}
m_{tan}=\lim_{\Delta x\to 0} \frac{f(x+\Delta x)-f(x)}{\Delta x}
\end{equation}

\section{Delta-Epsilon}
Let's say that for some function $f(x)$ that as $x\to c$, $f(x)\to L$.  That is
to say:
\begin{equation}
\lim_{x \to c} f(x) = L
\end{equation}

We can describe $\epsilon$ as a quantity that is the maximum distance the limit
$L$ can be away from the actual value (on a continuous function) at $f(c)$.

Furthemore, $\delta$ is the units away from $x$ to $c$ as we take the limit of
$f$.

Therefore, we can summarize these two statements and presume that for every
$\epsilon > 0$, there is a $\delta > 0$ such that for all $x$:
\begin{equation}
0 < |x-c| < \delta
\end{equation}

and...
\begin{equation}
|f(x)-L| < \epsilon
\end{equation}

\section{The Derivative}
The derivative of $f(x)$ is described as $\frac{d}{dx}f(x)$ and is itself a
function.  The derivative of $f$ is a function that for any $x$ produces a value
equal to the slope of the tangent line of $f$ at $x$.

One way to find the derivative of a function is to take the Newton Quotient of
function.  This is true because:
\begin{equation}
\frac{d}{dx}f(x)=\lim_{\Delta x\to 0} \frac{f(x+\Delta x)-f(x)}{\Delta x}
\end{equation}

\textit{Note}: since the derivative of a function and you can take the deriative
of any function, you can take the derivative of a function an arbitrary number
of times. Physics is the only real world application that requires you to take
the derivative of a function more than two times, but this class frequently
requires you to take the second derivative of a function.  This is often
expressed by the following notation:
\begin{equation}
\frac{d^2y}{dx^2}f(x)
\end{equation}

However, there are rules that we can apply to simplify the process of finding a
derivative.

\subsection{Derivation Rules}
\subsubsection{The Constant Rule}
\begin{equation}
\frac{d}{dx}c=0
\end{equation}

\subsubsection{The Constant Multiple Rule}
\begin{equation}
\frac{d}{dx}cf(x)=c\frac{d}{dx}f(x)
\end{equation}

\subsubsection{The Power Rule}
\begin{equation}
\frac{d}{dx}x^n=nx^{n-1}
\end{equation}

\subsubsection{The Sum and Difference Rule}
\begin{equation}
\frac{d}{dx}\big(f(x)\pm g(x)\big)=\frac{d}{dx}f(x) \pm \frac{d}{dx}g(x)
\end{equation}

\subsubsection{The Product Rule}
\begin{equation}
\frac{d}{dx}\big(f(x)g(x)\big)=f(x)\frac{d}{dx}g(x)+g(x)\frac{d}{dx}f(x)
\end{equation}

\subsubsection{The Quotient Rule}
\begin{equation}
\frac{d}{dx}\Bigg(\frac{f(x)}{g(x)}\Bigg)=\frac{f'(x)g(x)-g'(x)f(x)}{g^2(x)}
\end{equation}

\subsubsection{The Chain Rule}
\begin{equation}
\frac{d}{dx}f(g(x))=f'(g(x))g'(x)
\end{equation}

\section{Implicit Differentation}
Consider the function that defines the unit-circle: $x^2+y^2=1$.  How would you
go about differentiating it?  Sure, we know that $\frac{d}{dx}x^2=2x$, but what
does $\frac{d}{dx}y^2$ evaluate to?  Differentation is easy when the variable we
are differentating agrees with the variable that we are taking the differential
with respect to, but what happens when they don't?

To evaluate $\frac{d}{dx}y^2$, we preform the normal differentation of $a^b$, so
we're left with $2y$, but we still haven't solved the problem of the mismatched
variables.  The solution is simple: we simply multiply the quantity that we have
partially differentiated by the result of differentating the free-hanging
variable: $2y\frac{dy}{dx}$
