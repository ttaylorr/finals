\chapter{Exponential and Logarithmic Functions}
\section{Exponential Functions}
An exponential function is a defined as any function, $f(x)$ such that the
independent variable $x$ is bound up somewhere in the exponent.  $f(x)=3^x$ is
an example of such a function:

\graph{3^x}

This classification of functions have a few simple properties:

\begin{itemize}
  \item{They are defined for all $x\in(-\infty,\infty)$}
  \item{Their range is: $(0, \infty)$}
  \item{They have intercepts at $(0,1)$}
  \item{They are increasing on their entire domain}
  \item{The $x$-axis serves as a horizontal asymptote}
  \item{They are smooth and continuous}
\end{itemize}

\subsection{Interest Functions}
These types of functions have applications in a few different areas.  For us, we
will be primarily concerned with manually and continually compouding interest:

\begin{itemize}
  \item{For $n$ compoundings per year: $A=P(1+\frac{r}{n})^{nt}$}
  \item{For coninuous compounding: $A=Pe^{rt}$}
\end{itemize}

\section{Logarithmic Functions}
The parent function of the logarithmic family is defined as follows:
\begin{equation}
  f(x)=\log_ax
\end{equation}

\graph{ln(x)}

\subsection{Properties of Logarithms}
\begin{itemize}
  \item{$\log_a1=0$}
  \item{$\log_a=1$}
  \item{$\log_aa^x=x$}
  \item{If $\log_ax=\log_ay$ then $x=y$}
\end{itemize}

\section{Properties of Logarithms}
A logarithm of any base $a$ can be represented as a quotient of two other
logarithms of base $b$, according to the following (Change of Base rule)
\begin{equation}
  \log_ax=\frac{\log_bx}{\log_ba}
\end{equation}

The product property states that you can expand a logarithm that is the product
of two numbers $u$ and $v$ into two logarithms that are added together, as in
the following generalization:
\begin{equation}
  \log_a(uv)=\log_au+\log_av
\end{equation}

The quotient property states what is already implied, that a logarithm
represented as a quotient of two numbers $u$ and $v$ can be expanded equally as
a an expression of two logarithms which get subtracted, as in the following
generalization:
\begin{equation}
  \log_a(\frac{u}{v})=\log_au-\log_av
\end{equation}

The power property is implied from the product rule.  You can expand out a power
wrapped up in a logarithm into a scalar that is multiplied to the logarithmic
expression, as in the following generalization:
\begin{equation}
  \log_au^n=n\log_au
\end{equation}

\section{Solving Logarithmic Functions}
You can apply the rules that have been previously explained in order to solve
for expressions bound up in logarithmic or exponential expressions.  The genral
case for solving these types of problems is explained below:

\begin{enumerate}
  \item{Reduce all variables that are bound in exponential expressions to
    loosely hanging variables that are included in logarithmic expressions}
  \item{Apply the three rules that are stated above}
  \item{Expand}
  \item{Simplify}
\end{enumerate}
